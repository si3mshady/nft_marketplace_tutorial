
NEWSLETTERS
SUBSCRIBE

TOP STORIES

TOP VIDEOS
Best Workplaces
Awarding excellence in company culture

Regular Rate Deadline: January 14, 2022

Apply Now
PRODUCTIVITY
This 4-Step Plan Will Help You Reach Your Most Ambitious Goals in 2022

HEALTH CARE
Messenger RNA Took on Covid-19. This Startup Says That's Just the Beginning

PRODUCTIVITY
2 Easy Ways to Beat the Stress of Ridiculous End-of-Year Deadlines and Improve Your Focus

LEAD
Employees With This 1 Quality Are High Performers. Here's Why

TAKING CARE
6 Ways to Build a Collaborative Workplace Culture in 2022 and Beyond

OWNER'S MANUAL
Why Emotionally Intelligent People Refuse to 'Status Pivot' When Their Egos Get Threatened

TAKING CARE
3 Things You Can (and Should) Do Right Now to Prepare for Biden's Vaccine Mandate

NEWSLETTERS
SUBSCRIBE
MOST PRODUCTIVE ENTREPRENEURS
Elon Musk Shared a Profoundly Simple Productivity Hack That Just May Change Your LifeYou may not be running multiple companies, but you can learn a lot from the way Elon Musk splits his time.
BY JUSTIN BARISO, AUTHOR, EQ APPLIED
@JUSTINJBARISO
Elon Musk.
Elon Musk.
 Getty Images
To say Elon Musk has a lot on his plate is a huge understatement. The über-productive entrepreneur is known mostly for his work as CEO of Tesla, which for a short time overtook Toyota as the most valuable automotive company in the world.

Of course, running a company the size of Tesla is plenty challenging. But Musk is also CEO of space exploration company SpaceX, which has a stated mission of taking humans to Mars and "other destinations in the solar system."

And that's not all. Musk has also founded or currently plays major roles in a number of other companies, with the following goals:

Developing artificial intelligence that benefits all of humanity (OpenAI)
Developing brain-machine interfaces to connect humans and computers (Neuralink)
Solving the problem of "soul-destroying traffic" by means of a large network of underground tunnels (the Boring Company)
Producing clean energy via solar panels (SolarCity, a subsidiary of Tesla)
With his hands in the mix of so many companies, one might assume that Musk's knowledge about each is minimal--but that's far from the truth. Garrett Reisman, an engineer and former astronaut who worked for years with Musk after leaving NASA to join SpaceX, marveled at Musk's ability to stay in tune with minor details at the company.

"I mean, I've met a lot of super, super smart people," said Reisman in a recent interview. "But they're usually super super smart on one thing. And he's able to have conversations with our top engineers about the software and the most arcane aspects of that. And then he'll turn to our manufacturing engineers and have discussions about some really esoteric welding process for some crazy alloy."

He continued, "And he'll just go back and forth, and his ability to do that across all the different technologies that go into rockets and cars and everything else he does--that's what really impresses me."

So, how does Musk do it? How does he manage his time, so he can remain deeply involved with so many things?

Musk himself revealed a very simple yet effective productivity hack in an interview some years ago:

He splits his time so that he can basically focus on one company at a time.

Speaking with journalist Alison van Diggelen back in 2013, Musk described flying back and forth between Northern and Southern California, generally splitting his week between his two main companies in the following way:

Monday: SpaceX

Tuesday: Tesla

Wednesday: Tesla

Thursday: SpaceX

Friday: SpaceX

Saturday: Tesla

Sunday: Tesla or SpaceX

Musk indicated this schedule was simply an example and changes depending on what he's currently working on. But comments from Reisman were consistent with the narrative that Musk generally chooses to focus on one company before hopping on a plane and starting fresh the next morning with another.

Of course, few of us are trying to run multiple companies as Musk is, nor do we have access to a private jet. And to be clear, I don't advocate working seven days a week, especially if you have a family as I do. 

But I've found great value in separating different aspects of my work by days of the week, and even further by morning and afternoon.

This allows for huge productivity gains, as it allows you to separate time for brainstorming and creative work, along with meeting, helping, and collaborating with others. It also helps you to avoid losing time by constantly switching tasks or attempting to multitask.

Further, by keeping projects, client work, or major tasks separated by day, you can see how complex thoughts between different topics interrelate. This aids your ability to think critically and solve problems, allowing you to apply principles from one project or industry to another. 

Hacking the Hack
The key is not to try to do as much as Musk does in a week. Instead, decide what your priorities are and then adapt accordingly. And as much as possible, try to find a primary focus for each day.

For example, here's what a typical week for me might look like:

Monday: Major project 

Tuesday: Workshop prep (morning); workshop delivery or podcast interview (afternoon)

Wednesday: Writing (morning); meetings (afternoon)

Thursday: Major client work 

Friday: Catch up on emails (morning); family time (afternoon)

I prefer to use Mondays on my biggest and most important project. This gives me a chance to work head down with focused concentration, without distractions from others. (I try my best to avoid any meetings on Mondays, making that day the most productive.)

Additionally, for each day I usually schedule mornings as quiet time. This allows me to use my primary brainpower for creative work. If I need to do any speaking or have a meeting, I try to schedule these in the afternoon.

I then try to split up my other responsibilities accordingly:

Tuesdays are for collaborative work, like trainings, workshops, or podcasts.

Wednesday afternoons are for general meetings.

Thursdays are reserved for my biggest client.

And while I respond to important emails throughout the week, I use Friday mornings to catch up on emails that have slipped off my radar. This schedule allows me to take off Friday afternoons, when it's challenging to remain productive anyway. Then, I focus on what's most important to me--my family. 

You may not have multiple companies to run like Elon Musk, but I bet you feel you do. If so, take a page out of Musk's book, and use the days of the week to help you stay organized and focused.

Doing so changed my life, and it can change yours, too.

Get Inc.’s top posts straight to your inbox. Sign up here and you’ll receive Today’s Must Reads before each day is done.
JUN 29, 2020
Like this column? Sign up to subscribe to email alerts and you'll never miss a post.
The opinions expressed here by Inc.com columnists are their own, not those of Inc.com.
NEWSLETTER
Inc. This Morning
A daily news digest curated for those interested in entrepreneurship

Type your email
Sign Up
INC. TOP ARTICLES
8 Ways Your Business Startup Strategy Should Parallel the Military's
left arrow
right arrow
SPONSORED BUSINESS CONTENT
Dianomi logo

PRIVACY POLICY
NOTICE OF COLLECTION
DO NOT SELL MY DATA
AD VENDOR POLICY
TERMS OF USE
ADVERTISE
HELP CENTER
SITEMAP
COPYRIGHT 2021 MANSUETO VENTUREScopyright policy logo